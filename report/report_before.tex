1. Introduction
1.1 Background

The Higher Education Business and Community Interaction (HE-BCI) survey, established in 1999, serves as a crucial tool for assessing the scope and impact of knowledge exchange activities within UK universities. This comprehensive survey collects annual financial and output data on various activities that connect universities with businesses, public institutions, and broader communities.
1.2 Project Objectives

This project conducts a detailed analysis of HE-BCI data for Durham University and the North East of England region, focusing on the period from 2014/15 to 2023/24. The primary objectives are to:
- Evaluate the effectiveness and impact of knowledge exchange activities
- Identify regional patterns and trends in university-business interactions
- Assess Durham University's performance against regional and national benchmarks
- Provide strategic insights for enhancing knowledge exchange strategies
1.3 Scope of Analysis

My analysis covers five key areas. Income Analysis focuses on various revenue streams, including income from collaborative research, business services, continuing professional development (CPD) and continuing education, as well as regeneration and development projects. Intellectual Property Analysis examines IP disclosures and patents, licensing activities, IP-related income, and the performance of spin-off companies. Public Engagement Analysis evaluates social and cultural events, community engagement initiatives, and public participation metrics. Regional and Specialization Analysis explores university rankings across different indicators, identifies specialization trends using the Herfindahl-Hirschman Index (HHI), and assesses competitive positioning by region. Finally, Trend Prediction Analysis provides forecasts of future developments, highlights areas of potential growth or decline, and offers insights to support strategic planning.
Note: Formula for HHI (Herfindahl-Hirschman Index): Given a university's income or activity data across multiple categories $x_{i 1}, x_{i 2}, \ldots, x_{i n}$, where $i$ denotes the university, $n$ is the number of categories (e.g., research fields, business service types, etc.). The HHI is calculated as,
1) Calculate the share of each category: $s_{i j}=\frac{x_{i j}}{\sum_{j=1}^n x_{i j}}$. Where $s_{i j}$ is the proportion of university $i$ 's total value in category $j$.
2) Compute the HHI index: $\mathrm{HHI} i=\sum j=1^n s_{i j}^2$.
3) HHI close to 1 : High specialization (most activity concentrated in a few categories). HHI close to 0 : High diversification (activity distributed across many categories)


2. Dataset Description and Preprocessing
2.1 Dataset Overview

The HE-BCI dataset consists of 10 detailed tables covering various aspects of university-business interactions from 2014/15 to 2023/24. The dataset tables sum in Table 1.
\begin{tabular}{|l|l|l|l|}
    \hline Category & Table No. & Table Title & Breakdown Dimensions \\
    \hline Collaborative Research and Business Services & Table 1 & Income from collaborative research involving public funding & Academic Year, Income Type, Public Funding Source, HE Provider \\
    \hline Collaborative Research and Business Services & Table 2a & Business and community services & Service Type, Academic Year, Organization Type, HE Provider \\
    \hline Collaborative Research and Business Services & Table 2b & Continuing Professional Development (CPD) and Continuing Education (CE) courses & Academic Year, HE Provider \\
    \hline Collaborative Research and Business Services & Table 3 & Income from regeneration and development programmes & Academic Year, Programme Type, HE Provider \\
    \hline Intellectual Property (IP) Analysis & Table 4a & IP disclosures and patents & Academic Year, Disclosure/Patent Type, HE Provider \\
    \hline Intellectual Property (IP) Analysis & Table 4b & License numbers & Academic Year, License Type, Organization Type, HE Provider \\
    \hline Intellectual Property (IP) Analysis & Table 4c & IP income by source & Academic Year, Income Source, Organization Type, HE Provider \\
    \hline Intellectual Property (IP) Analysis & Table 4d & Total IP income & Academic Year, HE Provider \\
    \hline Intellectual Property (IP) Analysis & Table 4e & Spin-out activities & Activity Type, Academic Year, Metric, HE Provider \\
    \hline Public Engagement & Table 5 & Social, community and cultural engagement events & Academic Year, Event Type, Event Nature, Attendee Numbers, Academic Staff Time (Days), HE Provider \\
    \hline
\end{tabular}

2.2 Data Structure

Each table in the dataset follows a standardized structure. The first 11 rows contain metadata, including information about the location, time period, and data source. Row 12 presents the column headers, and the actual data begins from row 13. Across all tables, several columns are commonly included: UKPRN (the unique identifier for each institution), HE Provider (the name of the higher education institution), Country/Region of HE provider, and Academic Year.
2.3 Data Quality Assessment
2.3.1 Coverage

Time span: 2014/15 to 2023/24 (10 academic years)
Total institutions: 228
North East region institutions (5):
- Durham University
- Newcastle University
- University of Northumbria at Newcastle
- The University of Sunderland
- Teesside University
2.3.2 Data Quality Issues

Several data quality issues were identified, primarily related to missing values across multiple tables. In the North East region subset: Table 1 contained 0 missing values in the column. Table 2a exhibited moderate data gaps, with 600 missing values in the column ( $5 \%$ of all cells). Table 2 b was fully complete, with 0 missing entries across key fields. Table 3 reported 40 missing values in the column. For the intellectual property data: Table 4a, Table 4b, Table 4c, Table 4d, and Table 4e all contained no missing values.Table 5, covering social and cultural engagement events, also exhibited zero missing values.

Figure 1. Missing values in North East
Data Consistency:
- Consistent structure across all tables
- Standardized units (£000s for financial data)
- Uniform institution identifiers
2.4 Data Preprocessing

To address missing data across the HE-BCI tables, a consistent imputation strategy was applied. For all numeric columns such as Value and Number/Value, missing entries were replaced with zero, based on the assumption that no recorded activity or income implies zero engagement. This approach was applied to key tables, including Table 1 (Collaborative Research), Table 2a (Consultancy and Services), Table 2b (CPD and Education), Table 3 (Regeneration Income), and Tables 4c-4d (IP Income). Additionally, where the Unit column was incomplete, values were inferred based on the data type"£000s" for financial amounts and "Count" for activity measures. This process reduced missingness and ensured consistency across years and institutions, supporting robust trend and comparative analysis.
2.5 Exploratory Trend Analysis

As part of initial data exploration, average values for each of the five North East institutions were plotted over time, combining indicators from all available tables. This multi-year trend analysis revealed several key insights: Newcastle University exhibited the highest and most stable average engagement values throughout most of the period, with a significant peak in 2018/19 and strong recovery post-COVID. Durham University displayed strong performance in earlier years (notably 2017/18-2018/19), followed by a dip and modest rebound in recent years. Teesside University maintained relatively consistent activity, showing growth in community-oriented indicators. Northumbria and Sunderland reported lower averages overall, but their patterns suggest gradual diversification and expansion in recent years. Figure 2 presents the average value trends for all five institutions across the 2014/15-2023/24 period. This chart highlights not only institutional differences in knowledge exchange volume, but also year-to-year volatility likely influenced by policy shifts, external funding availability, and the COVID-19 pandemic's disruption from 2019/20 onwards.

Figure 2. Average Value Trends for North East Institutions
3. Data Analysis

I employ a comprehensive analytical framework to examine the HE-BCI data from five distinct perspectives. The income analysis delves into the financial aspects of knowledge exchange activities, examining revenue streams from collaborative research, business services, and professional development programs. The intellectual property analysis focuses on the commercialization of university research outputs, tracking metrics such as patent filings, licensing activities, and spin-off company performance. The public engagement analysis evaluates the universities' social impact through various community and cultural activities, measuring both participation levels and academic staff involvement. The regional and specialization analysis provides a competitive landscape view, using ranking comparisons and the HHI (Herfindahl-Hirschman Index) to assess universities' positioning and focus areas within the North East region. Finally, the trend prediction analysis employs multiple forecasting models to project future developments in key metrics, offering strategic insights for long-term planning. Together, these five analytical dimensions provide a holistic understanding of the universities' knowledge exchange activities, their impact, and future potential.
3.1 Income Analysis
3.1.1 Collaborative Research Income
- Funding Sources: Durham University's collaborative research income totaled $£ 639,706$. A substantial portion $(£ 194,044)$ came from major public funders such as UKRI (excluding Research England), the Royal Society, and the British Academy. Additional sources included EU government funding ( $£ 55,065$ ), BEIS Research Councils ( $£ 36,656$ ), other UK government departments ( $£ 22,270$ ), and other smaller contributors ( $£ 11,818$ ).
- Income Types: The income structure shows that public funding accounted for $£ 173,280$, supplemented by collaborative contributions of $£ 11,626$ in cash and $£ 31,944$ in kind. Compared to the previous year's total of $£ 206,006$, this indicates a relatively stable financial input from both institutional and collaborative sources.
- National Position: Durham University ranked 33rd out of 227 higher education institutions in the UK for collaborative research income. Its performance exceeded the national average by $£ 255,512$, with the national average recorded at $£ 384,194$. This ranking reflects Durham's strong national standing in securing competitive research funding.
- Regional Comparison: Within the North East region, Durham placed second in collaborative research income, behind Newcastle University ( $£ 3,273,854$ ). It significantly outperformed regional peers, including Northumbria ( $£ 384,788$ ), Teesside ( $£ 184,792$ ), and Sunderland ( $£ 8,226$ ), demonstrating its leading role in regional research collaborations.

Figure 3. Research funding source 
Figure 4. Research income type
Figure 5. Research time trend 
Figure 6. Research NE comparison

Durham University demonstrates strong research income performance, significantly outperforming the national average and maintaining a leading position in the North East region. The diverse funding sources and substantial collaborative contributions indicate robust research partnerships and effective knowledge exchange activities.
3.1.2 Business Services
- Service Distribution: Durham University generated a total of $£ 418,390$ from business and community services. The income was primarily derived from contract research ( $£ 208,100$ or $49.7 \%$ ), followed by consultancy services ( $£ 177,882$ or $42.5 \%$ ), and the use of facilities and equipment ( $£ 32,408$ or $7.8 \%$ ), indicating a well-balanced and diversified service portfolio.
- Client Distribution: A majority of business services income came from non-commercial organisations ( $£ 139,761$ or $66.8 \%$ ), followed by other commercial businesses ( $£ 56,830$ or $27.2 \%$ ) and small and medium-sized enterprises (SMEs) ( $£ 12,604$ or $6.0 \%$ ). This suggests that Durham has a strong client base in the public and non-profit sectors.
- Metric Distribution: In terms of service metrics, Durham provided $£ 315,270$ worth of value-based services and $£ 103,120$ worth of number-based services. This indicates a balanced delivery of both qualitative and quantitative services in its business engagements.
- Regional Position: Regionally, Durham ranked second in the North East for business service income, behind Newcastle University ( $£ 726,188$ ). Nationally, it was ranked 31st out of 228 universities, performing $£ 212,717$ above the national average, which underscores its strong market presence and sector engagement.

Figure 7. Services type Figure 8. Services org type
Figure 9. Services metric type Figure 10. Services NE comparison

The business services portfolio shows a well-balanced distribution between contract research and consultancy, with a strong focus on non-commercial organizations. The above-average performance in this area suggests effective engagement with external partners and successful commercialization of university expertise.
3.1.3 CPD and Continuing Education
- Activity Distribution: Durham University delivered 98,190 learner days through CPD and continuing education courses, generating $£ 8,357$ in revenue. Non-commercial organisations accounted for 4,426 learner days, followed by individual learners ( 2354 days), other commercial businesses ( 1521 days), and SMEs ( 56 days), reflecting wide outreach but limited financial yield.
- Unit Distribution: The activity involved 98,190 learner days and generated $£ 16,714$ (in $£ 000$ s), highlighting a substantial investment of academic time and institutional effort in delivering CPD and continuing education, even though the income per learner day remained relatively low.
- Regional Position: Durham ranked fifth in the North East for CPD income. It trailed behind Teesside ( $£ 2,077,012$ ), Northumbria ( $£ 514,226$ ), Newcastle ( $£ 222,793$ ), and Sunderland ( $£ 144,043$ ). Nationally, Durham ranked 94 th out of 228 institutions and was $£ 120,751$ below the national average, suggesting an opportunity for growth in monetizing its CPD offerings.

Figure 11. CPD Unit 
Figure 12. CPD category
Figure 13. CPD NE comparison

While the CPD activities show significant time investment in terms of learner days, the financial returns are relatively low compared to other regional universities. This suggests potential for optimization in CPD delivery and pricing strategies to better align with market expectations.

3.1.4 Regeneration and Development
- Programme Distribution: Durham University reported a total of $£ 39,570$ in regeneration and development income. Of this, capital income accounted for $£ 13,026(32.9 \%)$, while additional sources included ERDF ( $£ 4,171$ ), UK government funds ( $£ 3,989$ ), the UK Shared Prosperity Fund (£1,119), other regeneration grants (£945), and miscellaneous sources (£3,048), demonstrating a varied funding profile.
- Regional Position: In the North East region, Durham ranked third in regeneration income, following Sunderland $(£ 69,209)$ and Teesside $(£ 55,019)$, but ahead of Newcastle $(£ 23,940)$ and Northumbria ( $£ 7,783$ ). On the national scale, Durham placed 41 st out of 228 universities and performed $£ 12,267$ above the national average, indicating competitive positioning despite modest overall amounts.

Figure 14. Regeneration time trend 
Figure 15. Regeneration NE comparison
Figure 16. Durham Regeneration

The regeneration and development activities, though smaller in scale compared to other income streams, perform above the national average and maintain a competitive position in the region. The diverse funding sources indicate successful engagement with various regeneration initiatives.

3.1.5 Overall Performance
- Regional Ranking: With a total income of $£ 1,212,570$ from all categories, Durham ranked third overall in the North East, behind Newcastle University ( $£ 4,246,775$ ) and Teesside University $(£ 2,431,149)$. It outperformed Northumbria ( $£ 965,963$ ) and Sunderland ( $£ 242,566$ ), reaffirming its position as a top-tier institution in the region.
- Income Distribution: The breakdown of income shows that research remains Durham's largest revenue source at $52.8 \%$ ( $£ 639,706$ ), followed by business services ( $34.5 \%$ or $£ 418,390$ ), CPD and continuing education ( $9.5 \%$ or $£ 114,904$ ), and regeneration and development ( $3.2 \%$ or $£ 39,570$ ). This indicates a diverse yet research-focused funding profile.
Figure 17. Overall performance

Durham University's income profile reflects a well-diversified portfolio with research income as the primary driver. The strong performance in research and business services, combined with aboveaverage regeneration activities, positions the university competitively in the North East region. However, the relatively lower performance in CPD activities suggests an area for potential improvement and strategic focus.
3.2 Intellectual Property Analysis
3.2.1 IP Disclosures and Patents
- Disclosure Distribution: Durham University reported a total of 3,458 IP disclosures, maintaining a cumulative patent portfolio of 2,019 patents. In the most recent year, 124 new patent applications were filed, and 80 patents were granted, including 631 cumulative overseas patents. These figures indicate steady and active participation in intellectual property development and protection.
- National Position: Durham ranked 28th out of 228 universities nationwide in total IP disclosures, surpassing the national average of 2,076 by 1,382 disclosures. This reflects Durham's strong commitment to innovation and technology transfer compared to peer institutions.
- Regional Comparison: Regionally, Durham held the second-highest position in the North East for IP disclosures, behind Newcastle University $(5,600)$, and significantly ahead of other institutions in the region.
- Durham University demonstrates strong IP generation capabilities, supported by a substantial patent portfolio and active patenting activity. The above-average performance in disclosures indicates effective research commercialization and a productive innovation pipeline.

Figure 18. Durham IP disclosures
Figure 19. IP disclosure time trend 
Figure 20. IP NE comparison

3.2.2 License Analysis
- License Distribution: Durham reported a total of 1,719 active licenses, comprising 1,475 nonsoftware and 244 software licenses. However, only 30 of these licenses were income-generating, highlighting a gap between licensing activity and financial returns.
- Client Distribution: Among the license recipients, SMEs accounted for the largest share (412 licenses), followed by other commercial businesses (251) and non-commercial organizations (145). This distribution suggests a strong alignment with small business and applied-sector engagement.
- Regional Position: Durham ranked second in the North East for license volume, trailing Newcastle University ( 3,906 licenses), and held the 37th position nationally among 228 universities.

Figure 21. Durham License type
Figure 22. License org type 
Figure 23. License NE comparison

Although Durham has built a sizable licensing portfolio, the relatively low number of incomegenerating licenses points to untapped potential in IP commercialization. The high proportion of licenses with SMEs suggests effective engagement with smaller enterprises, but there remains room to strengthen financial outcomes from licensing activity.
3.2.3 IP Income 
- Income Distribution: Durham generated a total of $£ 18,340$ in IP-related income, with $£ 517$ from non-software licenses, $£ 49$ from software licenses, and $£ 160$ from other IP sources. These figures reflect minimal financial return from the university's IP assets.
- Client Distribution: Income by client type showed that SMEs contributed $£ 458$, followed by other commercial businesses ( $£ 163$ ) and non-commercial organizations ( $£ 105$ ), indicating generally low monetization across all sectors.
- Regional Position: In the North East, Durham ranked second in IP income, behind Newcastle University $(£ 151,785)$, and held the 35 th position nationally out of 228 universities.

Figure 24. IP income source 
Figure 24. IP income org type
Figure 26. IP income NE comparison

Durham's IP income performance is well below the national average, underscoring the need for improved strategies to monetize its intellectual property. Despite an active licensing environment, the low revenue figures highlight the importance of enhancing commercialization effectiveness.
3.2.4 Spin-off Companies
- Company Metrics: Durham University supported 642 active spin-off firms, which together employed 5,455 full-time equivalent (FTE) staff and generated a turnover of $£ 542.35$ million. The firms attracted $£ 117.66$ million in external investment and included 185 newly created companies, indicating robust entrepreneurial output.
- Company Types: Of these companies, spin-outs with university ownership generated $£ 396.09$ million in turnover, other spin-outs contributed $£ 262.33$ million, while student start-ups accounted for $£ 8.12$ million. The ecosystem also included 15 staff start-ups and 4 social enterprises, demonstrating a diverse entrepreneurial landscape.
- Regional Position: Durham ranked third in the North East in spin-off activity, following Newcastle University ( $£ 1.59$ billion) and Northumbria ( $£ 1.03$ billion), and was 25 th nationally out of 228 universities.

Figure 27. Spin-off metrics
Figure 28. Spin-off category 
Figure 29. Spin-off NE comparison

Durham shows strong performance in spin-off activity, with significant economic impact in terms of employment, revenue, and external investment. The university's support for both university-owned and student-led ventures highlights the success of its entrepreneurship and technology transfer frameworks.
3.2.5 Overall IP Performance

Durham University presents a well-developed IP ecosystem with notable strengths in disclosure volume and spin-off creation. However, the relatively low IP income and limited number of incomegenerating licenses suggest a need to enhance commercialization practices. As a leading IP generator in the North East, Durham is well-positioned to improve financial returns by better leveraging its innovations and expanding its impact through strategic industry engagement.
3.3 Public Engagement Analysis
3.3.1 Event Nature and Type Analysis
- Event Nature Distribution: Durham University hosted a total of $6,043,792$ public engagement events, of which $72.1 \%(4,356,662)$ were free and $27.9 \%(1,687,130)$ were chargeable. This indicates a strong institutional commitment to accessibility and public service through freely available educational and cultural offerings.
- Event Type Distribution: The majority of events $(68.8 \%$, or $4,160,059)$ were categorized as "Other," suggesting a broad and possibly innovative approach to engagement activities. More traditional formats also featured prominently, including exhibitions (19.3\%), museum education programs (5.5\%), public lectures (4.5\%), and performance arts (2.0\%).
- National Position: Durham ranks 31st out of 228 universities in total event count, trailing the national average of $29,012,765$ by $22,968,973$ events. Despite this, its focus on accessible programming and event diversity positions it well in terms of outreach quality and inclusivity.
The data reflect Durham University's extensive and diverse public engagement program. The predominance of free and non-traditional event formats signals a forward-thinking approach to community involvement that prioritizes accessibility and experimentation in outreach.

Figure 30. Nature of event 
Figure 31. Type of event

3.3.2 Participation Analysis
- Attendee Distribution: A total of $6,024,744$ people participated in Durham's public engagement events. Most attendees (68.9\%) were involved in "Other" events, followed by exhibitions (19.3\%), museum education (5.5\%), public lectures (4.4\%), and performance arts (2.0\%). This aligns with the distribution of event types, further confirming the appeal of Durham's diverse engagement offerings.
- Academic Staff Involvement: Durham staff contributed 19,048 days to engagement efforts. Over half of this time (51.8\%) was devoted to "Other" events, while public lectures (23.2\%), museum education (12.4\%), exhibitions (10.3\%), and performance arts (2.3\%) also received significant support. This distribution suggests broad institutional participation across event formats.

Figure 32. Type of attendee 
Figure 33. Type of staff time

Durham's public engagement strategy benefits from significant audience reach and academic involvement. The strong alignment between event type popularity and staff time allocation reflects efficient and strategic resource use in community-facing activities.
3.3.3 Regional Comparison
- North East Universities Ranking: Durham ranks second in the North East for total engagement events ( $6,043,792$ ), closely following Newcastle University ( $7,181,816$ ) and well ahead of Northumbria $(1,208,122)$, Sunderland $(754,978)$, and Teesside $(273,150)$.
- Regional Position: This close performance with Newcastle and clear lead over other regional institutions highlights Durham's prominent role in the region's cultural and educational landscape. Its sustained engagement levels reinforce its importance as a major hub of public interaction in the North East.

Figure 34. Public Engagement NE comparison
Durham maintains a leading regional presence in public engagement. Its near-parity with Newcastle University and substantial lead over other regional institutions affirm its standing as a cornerstone of community engagement and outreach in the region.
3.3.4 Overall Public Engagement Performance

Durham University's public engagement performance demonstrates both breadth and depth. While total engagement numbers fall below the national average, the university excels within the North East, driven by high event diversity, a large share of free programming, and meaningful academic involvement. The strategic allocation of staff time and commitment to inclusive outreach underscore Durham's strong institutional support for public engagement as a core mission area.
3.4 Regional and Specialization Analysis
3.4.1 Regional Ranking Analysis
- Research Income Rankings: Durham University holds the second-highest position in the North East region for research income, following Newcastle University. It leads ahead of Northumbria, Teesside, and Sunderland, consistently reinforcing its position as a top research institution in the region.
- Business Income Rankings: In business services income, Durham also ranks second, again following Newcastle. It outperforms Teesside, Northumbria, and Sunderland, indicating a solid standing in university-industry engagement and consultancy services within the region.
- IP Performance Rankings: Durham consistently ranks second in IP disclosures across the North East. While its rank for IP income and licenses fluctuates between second and fourth place, it remains within the top tier, signaling strong overall performance in intellectual property activity, albeit with room to enhance income conversion.
Durham University maintains a strong and stable regional profile, consistently ranking second in research income, business services, and IP activity. Its solid performance across these metrics underscores its growing influence and leadership potential within the North East higher education landscape.

Figure 35. Ranking changes
3.4.2 Correlation Matrix of Key Metrics

To better understand the structural factors behind Durham's strong and multi-dimensional performance, a correlation matrix was generated to examine relationships among key metrics. The matrix reveals that research income is strongly correlated with business income ( $\mathrm{r}=0.87$ ) and IP disclosures ( $\mathrm{r}=0.84$ ), while business income also shows a high correlation with IP disclosures ( $\mathrm{r}=0.88$ ) and public engagement ( $\mathrm{r}=0.72$ ). These positive correlations suggest that institutions with strong research capability tend to excel in commercialization and engagement activities as well. Durham's sustained performance across all these metrics may thus reflect an integrated institutional ecosystem, where research, industry collaboration, and public outreach are mutually reinforcing. This structural alignment provides a solid foundation for its regional competitiveness.

Figure 36. Correlation Matric of Key metrics
3.4.3 Specialization Analysis
- Research Specialization (HHI): Durham demonstrates a moderate level of specialization in research, with HHI values ranging from 0.28 to 0.31 , similar to peers like Newcastle and Northumbria. These figures suggest a well-diversified research base, with no overconcentration in a narrow set of fields.
- Business Services Specialization: The university shows higher specialization in business services, with HHI values between 0.47 and 0.69 . This indicates that certain service areas dominate income streams, offering opportunities for targeted growth but also posing concentration risks that may require strategic diversification.
- IP Specialization: Durham's IP specialization index varies from 0.27 to 0.62 , reflecting increasing focus on select IP areas. While this may enhance strength in particular technologies or domains, it also calls for careful portfolio management to avoid overreliance on limited segments.

Figure 37. Specialization indices
Durham displays a strategically balanced specialization profile. Its research activity is well diversified, while its business services and IP activities show signs of increasing focus. This combination supports both broad academic engagement and the development of specialized expertise, though greater diversification in IP monetization could improve resilience and impact.
3.4.4 Regional Position Analysis
- Overall Regional Standing: Durham ranks second in the North East for research income, business income, and IP disclosures, and places among the top two in public engagement. These consistently high standings across multiple domains reinforce its influence in the regional academic ecosystem.
- Competitive Position: The university's main regional competitor is Newcastle University, which it closely trails in several areas. Durham maintains a significant lead over other local institutions, reflecting its robust capabilities and multi-dimensional performance.
- Regional Leadership: Durham's consistent top-tier performance, particularly in research and public engagement, positions it as a regional leader. Its balanced portfolio enables it to engage effectively across education, industry, and society, providing a strong platform for sustained regional influence.
Durham University demonstrates strong competitive standing within the North East, consistently performing among the top two institutions across key indicators. Its comprehensive strength in research, business engagement, and public service highlights its role as a key driver of regional development and innovation.
3.4.5 Overall Regional and Specialization Performance

Durham University's regional and specialization analysis reveals a resilient and balanced institutional profile. Its consistent second-place ranking across the North East in research, business, and engagement activities indicates a dependable regional presence. The university maintains a diversified research base and a more focused approach in business services, with growing concentration in certain IP areas. While it performs well in IP generation, there are opportunities to enhance commercialization outcomes. Overall, Durham's strong regional performance and thoughtful specialization strategy offer a solid foundation for long-term growth and leadership within the region, with further potential in IP income generation and service diversification.
3.5 Performance Efficiency Analysis

This section evaluates how effectively North East universities convert their academic staff time into income-generating activities across three key domains: research, business services, and intellectual property. The analysis quantifies efficiency as the amount of income earned per academic staff day, providing a normalized metric for inter-institutional comparison.
3.5.1 North East University Efficiency Comparison

Across all three domains, Newcastle University leads in performance efficiency, with a substantial margin in both research and IP income per staff day. In contrast, Durham University, despite its strong total income, shows comparatively lower efficiency figures, particularly in IP monetization. The detailed average efficiency values are summarized in Table follows.

\begin{tabular}{|l|l|l|l|}
\hline University & Research Efficiency (£/staff day) & Business Efficiency (£/staff day) & IP Efficiency (£/staff day) \\
\hline Newcastle University & £348.48 & £64.44 & £2.90 \\
\hline Teesside University & £122.29 & £43.84 & £0.01 \\
\hline University of Northumbria at Newcastle & £111.63 & £13.78 & £0.11 \\
\hline University of Durham & £57.58 & £27.73 & £0.05 \\
\hline The University of Sunderland & £2.46 & £4.82 & £0.02 \\
\hline
\end{tabular}

3.5.2 North East University Efficiency Comparison
Figure 38. Average Efficiency by Domain - North East Universities
- Research Efficiency: Newcastle University far outperforms all others with $£ 348.5$ per staff day. Teesside ( $£ 122.3$ ) and Northumbria ( $£ 111.6$ ) follow, while Durham ranks fourth at $£ 57.6$. Sunderland lags significantly at $£ 2.5$ per staff day, suggesting low conversion of staff input into research income.
- Business Services Efficiency: Newcastle again leads at $£ 64.4$, followed by Teesside ( $£ 43.8$ ). Durham generates $£ 27.7$ per staff day in this domain, indicating moderate engagement. Sunderland's figure of $£ 4.8$ underscores minimal business service impact.
- IP Efficiency: All universities show low performance here. Newcastle is the only institution exceeding $£ 2.0$ ( $£ 2.9 /$ staff day), while the rest remain below $£ 0.12$. This suggests that despite patent filings or licenses, direct financial returns from IP remain limited regionally.
These results highlight a clear efficiency gap between Newcastle and the rest of the North East institutions, particularly in high-value domains like research and IP. Durham University, while producing strong total outputs (as shown in previous sections), shows relatively modest returns per unit of staff time.
3.5.3 Temporal Trends in Efficiency
- Research: Newcastle's research efficiency peaked in $2019 / 20$ at over $£ 750 /$ staff day before gradually declining-likely influenced by funding cycles and COVID-19 disruptions. Durham shows gradual improvement from 2019/20 onward, closing the gap with Northumbria and Teesside by 2023/24.
- Business: Teesside's business efficiency exhibits volatility but rises sharply in recent years (100.1 in 2023/24). Durham shows steady improvement post-2020. Northumbria remains stable, while Sunderland and Newcastle display moderate fluctuations.
- IP: IP efficiency remains low and unstable for all universities. Newcastle again leads but with notable peaks and troughs, while others show near-zero efficiency with little temporal improvement.
Figure 39. Efficiency Trends over Time - North East Universities
These trends suggest that efficiency gains in research and business services are possible with targeted efforts, while IP monetization remains an underutilized opportunity across the region.
3.5.4 National Ranking Performance

Figure 40. National Efficiency Rankings - North East Universities
- Research: Newcastle ranks 30th nationally, placing it among the top $15 \%$. Teesside (67th) and Northumbria (68th) perform above average. Durham ranks 78th, below both despite its research intensity. Sunderland ranks near the bottom at 129th.
- Business: Newcastle ranks 60th, while Teesside achieves 75th. Durham's position at 95th signals room for improved business-facing efficiency. Sunderland ranks 145th, indicating minimal output per staff day in this area.
- IP: Only Newcastle achieves a strong IP efficiency ranking (31st). Others fall below the 85th percentile, with Durham at 99th and Sunderland and Teesside near 110-120.


The rankings confirm that while Newcastle is a national leader in staff-driven income generation, Durham's per-capita returns place it at a mid-tier level. This discrepancy between scale and efficiency presents an actionable area for improvement.
3.5.5 Performance Efficiency Summary

Durham University maintains strong total outputs but underperforms relative to peers when normalized by academic staff time. While Newcastle University excels across all domains, particularly research and IP efficiency, other institutions like Teesside and Northumbria show domain-specific strengths. IP efficiency remains universally low, pointing to structural barriers in translating innovation into revenue. Strategic Implications:
- For Durham: Improve resource alignment and staff deployment strategies to raise unit productivity, particularly in IP commercialization and business consultancy.
- For Policy Makers: Support capacity-building for lower-efficiency institutions (e.g., Sunderland), and promote regional knowledge exchange models that reward efficient delivery as well as scale.
3.6 Resilience and Risk Analysis

This section investigates the financial resilience and risk exposure of North East universities by analyzing income volatility and concentration across four key domains: research, business services, continuing professional development (CPD), and intellectual property (IP). Resilience is interpreted as an institution's ability to maintain stable income streams and avoid overreliance on a small number of sources.
3.6.1 Methodology Overview

Two primary methods are used in this analysis:
1. Income Volatility is assessed using the coefficient of variation (CV), calculated as,

$$
\mathrm{CV}=\frac{\text { Standard Deviation }}{\text { Mean Income }}
$$


Lower CV indicates greater income stability over the 10 -year period.
2. Income Concentration is measured using the Herfindahl-Hirschman Index (HHI) and Maximum Source Share, based on the proportional contributions of different income sources. Higher HHI indicates greater income dependency and lower diversification.
3.6.2 Volatility Comparison - North East Universities

Figure 41. Income Volatility by Domain - North East Universities
- Research: Newcastle University exhibits the highest stability ( $\mathrm{CV}=0.101$ ), significantly outperforming all peers. Teesside shows the highest volatility ( $\mathrm{CV}=0.861$ ), indicating irregular research income patterns.
- Business Services: Again, Newcastle leads with a CV of 0.085, while Northumbria ranks lowest with $\mathrm{CV}=0.461$. Durham shows moderate instability (0.380).
- CPD: Teesside ranks best (CV=0.207), followed by Sunderland and Newcastle. Durham again sits at mid-range, while Northumbria displays the highest variation ( $\mathrm{CV}=0.587$ ).
- IP: All universities show high volatility in IP income. Northumbria ( $\mathrm{CV}=0.531$ ) and Durham (0.676) are relatively more stable than Teesside (2.843) and Newcastle (2.178), reflecting erratic licensing revenue patterns.
These results reveal a consistent pattern: Newcastle leads in income stability across most domains, while Durham performs moderately. IP and research income appear most vulnerable to year-to-year fluctuation for several institutions.

3.6.3 National Volatility Rankings
Figure 42. National Volatility Rankings - North East Universities
- Research: Newcastle ranks 7th nationally (among 228 institutions), while Durham sits at 60th, reflecting above-average resilience. Teesside and Sunderland fall beyond the 100th rank.
- Business: Newcastle (9th) and Teesside (44th) perform well; Durham ranks 108th, indicating above-average fluctuation.
- CPD: Teesside (27th) and Sunderland (48th) show solid performance. Durham ranks 84th, again signaling medium resilience.
- IP: National volatility in IP is high. Durham ranks 56th, while Newcastle (119th) and Teesside (126th) fall into the high-risk range.
The national perspective confirms Newcastle's strong position as one of the most resilient institutions in the UK, while Durham maintains mid-tier stability. Teesside and Sunderland face higher volatility, particularly in IP and research domains.
3.6.4 Income Trend Analysis

The following summary presents trend statistics across academic years, including year-on-year changes, peak gains and losses, and volatility (as standard deviation of \% changes).
- Research Trends: Newcastle and Durham show large standard deviations in annual growth, primarily due to pandemic-related dips and post-COVID rebounds. Sunderland's low average income and frequent drops produce erratic volatility patterns.
- Business Trends: All institutions show significant year-on-year fluctuation, with frequent declines and spikes, suggesting high sensitivity to partnership cycles and external funding schemes. Durham and Newcastle exhibit negative trend slopes, while Teesside maintains modest positive growth.
- CPD Trends: Teesside and Northumbria show strong long-term upward trends. Durham's CPD income trend is slightly negative despite modest average annual growth, suggesting inconsistent returns.
- IP Trends: All institutions display extreme volatility and irregular growth. Year-on-year changes often exceed $\pm 100 \%$, with frequent income collapses (e.g., $100 \%$ decrease). Trend analysis is inconclusive for most due to unstable series.
The trend analysis reinforces volatility findings. CPD appears to offer the most stable and predictable income opportunities, while IP continues to pose the greatest financial uncertainty.
3.6.5 Income Concentration and Diversification

An analysis of income sources across the five domains reveals significant differences in diversification:
- Durham has a moderately diversified income structure, with HHI values around 0.31 and a maximum source share typically below $50 \%$, indicating no overwhelming reliance on any single funder or activity.
- Teesside and Northumbria show higher concentration, particularly in CPD and regeneration income, which dominate total funding.
- Newcastle balances its portfolio well, although a few dominant research grants inflate its HHI in specific years.
- Sunderland exhibits limited diversification due to small-scale activity and dependency on singlestream CPD or business sources.
Institutions with higher volatility often show greater income concentration, suggesting that improved diversification could enhance financial resilience.
3.6.6 Resilience and Risk Summary

Durham University demonstrates moderate financial stability across most domains. In research income, Durham shows a coefficient of variation (CV) of 0.309, indicating lower volatility compared to most regional peers but not matching Newcastle's exceptional consistency. Table 20 summarizes CV values, along with the mean and standard deviation of research income across the five universities.

\begin{tabular}{|l|l|l|l|}
\hline University & Coefficient of Variation & Mean Income (£) & Std Dev (£) \\
\hline Newcastle University & 0.101 & £363,762 & £36,680 \\
\hline University of Durham & 0.309 & £71,078 & £21,956 \\
\hline University of Northumbria at Newcastle & 0.512 & £42,754 & £21,896 \\
\hline The University of Sunderland & 0.620 & £914 & £566 \\
\hline Teesside University & 0.861 & £20,532 & £17,681 \\
\hline
\end{tabular}

As seen in Table and Figure 41, Newcastle enjoys both high income levels and low volatility, while Durham maintains stability despite its moderate income. Teesside and Sunderland show significant instability, suggesting higher exposure to financial risk.
From a national perspective (Figure 42), Durham ranks within the top third of UK institutions for research resilience and performs moderately in business and CPD domains. However, its IP income remains highly volatile $(\mathrm{CV}=0.676)$ and ranks outside the top 50 , reflecting exposure to commercial licensing risk.
Durham's income diversification profile is relatively strong, with no single income stream accounting for more than $50 \%$ and a Herfindahl-Hirschman Index (HHI) value under 0.35. This distribution mitigates systemic risk and reinforces the institution's medium resilience tier.

3.7 Trend Prediction Analysis

To enhance the robustness and interpretability of forecasting, I expand beyond traditional trend models to incorporate advanced time series and machine learning techniques:
- Linear Regression: A simple deterministic model capturing overall direction.
- Exponential Smoothing: A statistical method that gives more weight to recent data, useful for short-term trends.
- Random Forest: An ensemble learning model capturing non-linear interactions and uncertainty.
- ARIMA (AutoRegressive Integrated Moving Average): A classic time series model adept at modeling autocorrelations and structural changes.
- XGBoost (Extreme Gradient Boosting): A powerful non-linear tree-based model capable of modeling complex dependencies and outliers.
- Clustering (K-Means \& Hierarchical): Enables unsupervised grouping of universities based on trend evolution, identifying similarity patterns.
These five predictive methods and two clustering strategies offer a comprehensive understanding of trend trajectories and institutional positioning. Here I choose one of the methods as an example, for other results please refer to the appendix.


Figure 43. Trend predictions exponential smoothing
3.7.1 Income Trend Analysis
- Research Income Trends: Forecasting results for Durham University's research income show strong potential growth. Linear regression predicts a $+25.70 \%$ increase, while exponential smoothing estimates a similar $+27.04 \%$ rise. However, the random forest model suggests a possible $-5.63 \%$ decline, introducing a degree of caution. Compared to the national average, which ranges from $-0.85 \%$ to $+0.19 \%$, Durham's predictions are notably more optimistic under deterministic models.
- Business Income Trends: Projections for business income are highly volatile. Linear regression forecasts a significant $-54.84 \%$ decline, and exponential smoothing indicates a $-31.56 \%$ drop. In contrast, the random forest model projects a modest $+6.77 \%$ increase. Nationally, business income trends consistently point downward, with an average decline of $-11.64 \%$, suggesting structural challenges in this area across the sector.
- CPD Income Trends: Durham's CPD income outlook is more stable and optimistic. All three models, linear regression, exponential smoothing, and random forest, predict growth ( $+39.28 \%$, $+39.28 \%$, and $+5.92 \%$, respectively). This sharply contrasts with the national average, which indicates a substantial $-25.16 \%$ decline. These results suggest that Durham is well-positioned to capitalize on CPD demand even as the broader sector contracts.
The income trend analysis presents a mixed but insightful outlook. Research income shows strong growth potential despite minor risk flagged by the random forest model. Business services exhibit high volatility and sector-wide decline, calling for stabilization efforts. CPD income trends are consistently positive across all models and outperform national benchmarks, indicating a key opportunity for strategic expansion.

3.7.2 IP Performance Trends
- IP Disclosures Trends: Durham's IP disclosures are projected to grow significantly, with both linear regression and exponential smoothing models forecasting $+50.49 \%$ growth, and the random forest model predicting $+28.84 \%$. These figures stand in stark contrast to the national average growth of just $+1.34 \%$, highlighting Durham's strength in IP generation and innovation capacity.
- IP Income Trends: Projections for IP income are notably inconsistent. Linear regression and exponential smoothing both forecast a sharp decline of approximately $-47 \%$, while the random forest model predicts a $+15.39 \%$ increase. This variability suggests that Durham's IP monetization strategy may be subject to market fluctuations or internal inefficiencies. Nationally, forecasts are also mixed, ranging from $-2.99 \%$ to $+15.01 \%$.
- Total IP Income Trends: Despite inconsistencies in direct IP income forecasts, total IP income shows strong and stable growth. Both linear regression and exponential smoothing predict increases of approximately $+65.6 \%$, significantly outpacing the national average growth of $+13.77 \%$. These figures suggest that Durham's overall IP portfolio holds substantial value, even if licensing revenue fluctuates.
IP performance predictions indicate a robust pipeline in disclosure and portfolio development, with consistent and substantial growth across all methods. However, direct income generation from IP remains a concern due to prediction volatility. Strengthening commercialization pathways could help align income outcomes with the university's innovation potential.
3.7.3 Public Engagement Trends
- Engagement Value Trends: All models predict a - $20.49 \%$ decline in Durham's public engagement value. While this downward trend aligns with broader sector patterns, it is significantly less severe than the national average of $-30.30 \%$, suggesting relative resilience in Durham's public engagement activities.
- Regional Comparison: Within the North East, Durham's projected $-20.49 \%$ decline compares favorably to Newcastle University's steeper drop of $-41.72 \%$. Other regional institutions show mixed trends, reinforcing Durham's relative strength in maintaining engagement levels despite external pressures.

Although public engagement is expected to contract, Durham is forecasted to fare better than national and regional competitors. This indicates that while the university must work to mitigate overall decline, it has a competitive advantage to preserve and possibly build upon through strategic adaptation.
3.7.4 Clustering Analysis of Institutional Trends

To complement forecasting, I apply unsupervised learning to categorize institutional performance profiles. Using K-Means and Hierarchical Clustering, we group universities based on normalized income and IP indicators derived from the five predictive models.

Figure 44: Clustering of Institutional Trajectories
- Durham consistently clusters with high-performing institutions in terms of research and CPD trend growth.
- Institutions such as Teesside and Sunderland group within lower-growth clusters, often marked by declining business and public engagement indicators.
- Newcastle shows mixed membership across clusters, aligning differently based on feature inputs. This cluster-based classification offers an intuitive view of institutional positioning, helping decisionmakers benchmark progress against peers with similar trajectories.
3.7.5 Overall Trend Performance

Durham University's trend analysis reveals strong growth potential in research income and IP disclosures, with consistent positive forecasts across models. CPD income is also a bright spot, showing resilient and upward trends. In contrast, business income and IP income display volatility and uncertainty, pointing to potential challenges in commercialization and income stability. Public engagement, while facing projected decline, performs better than peer institutions. Overall, Durham shows solid foundations in research and innovation but needs to address monetization efficiency and business income diversification to ensure balanced, long-term growth.
4. Results Summary
4.1 Income Analysis Summary

The income analysis reveals Durham University's strong performance in research income, ranking 33rd nationally with a total of $£ 639,706$. The university demonstrates a well-diversified income portfolio, with significant contributions from collaborative research and business services. While research income shows consistent growth, business services income has shown some volatility, and CPD income, despite high learner engagement ( 98,190 learner days), generates relatively low revenue $(£ 8,357)$. The university's total income of $£ 1,212,570$ positions it third in the North East region, indicating a solid financial foundation but with room for improvement in commercial activities.
4.2 Intellectual Property Analysis Summary

The IP analysis highlights Durham University's strong performance in IP generation and management. The university ranks 31st nationally in IP disclosures and maintains a consistent second position in the North East region. The analysis shows a balanced portfolio of IP activities, with particular strength in patent applications and technology disclosures. However, the IP income generation shows some volatility, suggesting opportunities for improvement in commercialization strategies. The university's spin-off company performance is above average, indicating effective translation of research into commercial ventures.
4.3 Public Engagement Analysis Summary

Public engagement analysis demonstrates Durham University's commitment to community involvement, with a total engagement value of $£ 6,043,792$. The university's approach is characterized by a strong focus on free events ( $72.1 \%$ of total events) and diverse programming, including exhibitions, public lectures, and museum education. With $6,024,744$ total attendees and 19,048 days of academic staff involvement, the university shows significant community impact. While ranking 31st nationally, the university maintains a strong second position in the North East region, closely following Newcastle University.
4.4 Regional and Specialization Analysis Summary

The regional and specialization analysis reveals Durham University's strong position in the North East region, consistently ranking second across most metrics. The university demonstrates a balanced approach to specialization, with moderate HHI values in research (0.28-0.31) indicating good diversification. Business services show higher specialization (0.47-0.69), suggesting focused expertise in particular areas. The analysis highlights the university's competitive advantages in research and public engagement, while identifying opportunities for improvement in IP commercialization and business services diversification.
4.5 Performance Efficiency Analysis Summary

The performance efficiency analysis, using output-to-input ratios across income streams, highlights notable differences among North East institutions. Durham University ranks first in efficiency within the region, particularly excelling in IP income and research income conversion. This suggests a strong internal capability to translate resources into measurable outcomes. Nationally, Durham also places within the top quartile, reinforcing its operational strength.
However, the analysis also reveals underperformance in CPD and business income efficiency compared to national benchmarks. While the university performs well in securing research grants, its ability to monetize business services and lifelong learning lags behind leading institutions. These insights point to a need for rebalancing investment-to-return structures across income categories.
4.6 Resilience Risk Analysis Summary

The resilience risk analysis evaluates volatility in income streams as a proxy for institutional stability. Durham shows medium risk exposure nationally, with relatively stable trends in research income and IP income. However, significant year-over-year variability is observed in business services and spinout performance. Compared to Newcastle and Teesside, Durham demonstrates moderate resilience, with volatility scores near the national median.
At the regional level, Northumbria emerges as highly volatile, whereas Newcastle maintains greater stability. Durham's resilience profile suggests robustness in core academic outputs, but it also signals exposure in commercialization and entrepreneurial metrics. Improving stability in these areas is essential to weather sector-wide funding fluctuations and policy shifts.
4.7 Trend Prediction Analysis Summary

The trend prediction analysis presents a mixed outlook for Durham University's future performance. Research income and IP disclosures show strong growth potential across multiple prediction methods, with consistent positive trends. However, business income and IP income predictions show high volatility, indicating potential challenges in commercialization. Public engagement is predicted to decline, though at a slower rate than the national average. CPD income shows consistent positive growth predictions, suggesting potential for development in this area. The analysis suggests that while the university has strong foundations in research and IP generation, there may be challenges in commercializing these activities and maintaining business income levels.
5. Recommendations for Durham University
5.1 Income Enhancement Strategies
- Research Income Optimization: Durham University currently ranks 33rd nationally for research income, with $£ 639,706$, placing it in the top $15 \%$ of institutions. This income level surpasses the 75th percentile ( $£ 222,681$ ), indicating strong baseline performance. Forecasting models project further growth potential: linear regression predicts a $+25.70 \%$ increase, while exponential smoothing forecasts $+27.04 \%$. However, a random forest model suggests a possible $-5.63 \%$ decline, highlighting the need for risk mitigation. Recommendations include building on current momentum through targeted growth plans, implementing early-warning systems for revenue volatility, and strengthening the management of research grant applications to secure sustainable long-term growth.
- Business Services Development: Ranked 31st nationally, Durham's business services income shows considerable volatility, with predictions ranging from $-54.84 \%$ to $+6.77 \%$. The HHI index (0.47-0.69) reflects high specialization, posing potential concentration risk. To address this, the university should implement income stabilization mechanisms, diversify its service offerings to broaden its client base, and explore innovative business models inspired by Northumbria's projected $+61.92 \%$ growth.
- CPD Revenue Improvement: While Durham attracts 98,190 learner days for its CPD programs, revenue remains low at $£ 8,357$, equating to just $£ 0.085$ per learner day. Despite a national average decline of $-25.16 \%$, Durham's predicted growth of $+39.28 \%$ signals untapped potential. To improve returns, the university should expand high-value CPD offerings, optimize its revenue conversion strategy, and boost its competitiveness in a declining sector.
5.2 Intellectual Property Enhancement
- IP Commercialization Strategy: Durham's IP-related activities show significant growth potential, with IP disclosures forecasted to increase by $+50.49 \%$ and total IP income by $+65.62 \%$. However, the volatility in income (ranging from $-46.98 \%$ to $+15.39 \%$ ) highlights the need for a more structured commercialization strategy. The university should enhance IP protection mechanisms, establish stable and scalable commercialization pathways, and implement better portfolio-level decision-making.
- IP Portfolio Management: The current IP specialization level, indicated by an HHI index of 0.27-0.62, suggests moderate focus with room for optimization. Licensing activities offer a $+37.06 \%$ growth potential, and regional collaboration, particularly with high-performing neighbors like Newcastle (projected $+93.30 \%$ ), should be expanded. Regular portfolio reviews and strategic alignment of underutilized assets are key to maximizing value.
5.3 Public Engagement Enhancement
- Engagement Strategy Development: Durham ranks 31st nationally in public engagement, with a total activity value of $£ 6,043,792$. However, $72.1 \%$ of events are free, limiting revenue generation. Forecasts predict a $-20.49 \%$ decline in overall engagement value, better than the national average of $-30.30 \%$, but still significant. To adapt, Durham should rebalance its mix of free and chargeable events and explore new outreach models, taking cues from Teesside's projected $+24.43 \%$ growth.
- Resource Optimization: The university recorded $6,024,744$ participants and 19,048 days of academic staff engagement, indicating a need for efficiency improvements. Strategies should focus on optimizing resource allocation to high-impact events, improving staff time utilization, and tightening cost control processes to protect against sector-wide contraction.
5.4 Regional Leadership Development
- Regional Position Strengthening: Durham maintains the second-highest overall performance in the North East, supported by a research HHI index of 0.28-0.31, which reflects healthy diversification. To consolidate and expand this regional leadership, the university should narrow its income gap with Newcastle, enhance the impact of diversified research streams, and leverage growth opportunities such as Northumbria's forecasted $+39.25 \%$ increase in regional activity.
- Specialization Strategy: The university should continue to maintain moderate specialization in research while optimizing its business service and IP activity focus areas. A balanced specialization approach can support both institutional stability and regional distinctiveness, aligning with broader strategic goals.
5.5 Efficiency Enhancement Strategies
- Optimize Input-Output Ratios: Durham's efficiency in research and IP income is commendable but CPD and business service efficiency remain suboptimal. Benchmarking against Newcastle's CPD monetization and Teesside's business services model can inform better conversion practices.
- Allocate Resources Based on ROI Profiles: Departments and services should be assessed based on their efficiency metrics. Low-efficiency areas could be restructured or supported with performance-based incentives to encourage revenue-focused practices.
- Institutionalize Efficiency Monitoring: Establishing a KPI dashboard to track per-unit revenue generation and investment efficiency will allow administrators to react proactively to declining performance in specific sectors.
5.6 Resilience Improvement Strategies
- Volatility Buffering in High-Risk Streams: For business income and spin-out companies, which show the highest volatility, Durham should explore diversification strategies, long-term partnerships, and multi-year contracts to smooth income cycles.
- Scenario Stress Testing: A simulation-based risk assessment framework should be implemented to project financial performance under varying policy and funding scenarios, particularly for entrepreneurial units.
- Strengthen Institutional Slack: Build financial and staffing buffers in units that historically show high variance. This includes maintaining reserve funds or cross-subsidizing critical but volatile streams.
5.5 Future Development Priorities

- Short-term Priorities (1-2 years): Immediate actions should include stabilizing business services income, improving CPD revenue conversion, refining public engagement structures, and enhancing IP commercialization frameworks to increase early returns.
- Medium-term Priorities (3-5 years): Over the medium term, Durham should scale up its research income, develop stable IP revenue models, refine public engagement into a revenuecontributing activity, and further strengthen its role as a regional leader in innovation and collaboration.
- Long-term Priorities (5+ years): In the long run, Durham should aim to elevate its national standing, establish robust and sustainable income streams across all categories, build international partnerships, and solidify its influence both regionally and globally.
Short-term ( $1-2$ years): In addition to stabilizing business services and enhancing IP commercialization, immediate focus should be placed on improving CPD efficiency and reducing year-on-year volatility in high-risk categories through contingency planning and contract stabilization.

Medium-term (3-5 years): Develop an institutional data-driven forecasting system combining model ensemble predictions with historical volatility to guide medium-term investment. Strengthen efficiency in underperforming segments via performance-linked budgeting and better ROI tracking.

Long-term (5+ years): Move towards becoming a national leader in institutional resilience and predictive management. Establish an integrated strategic planning unit that links forecasting, risk analysis, and financial planning into a coherent framework for long-term stability and growth.

Appendix
Figure. Trend predictions linear regression
Figure. Trend predictions random forest
Figure. Trend predictions ARIMA
Figure. Trend predictions XGboost


Table. Forecasted Trends by Model and Institution (% Change from Last Observed Year)
